\section{Câu hỏi}
    \begin{enumerate}
        \item Xạ trị trong (xạ trị áp sát) là gì? Quy trình xạ trị trong suất liều cao (các bước thựciện)
        \item Xạ trị ngoài (xạ trị từ xa) là gì? Quy trình xạ trị ngoài bằng máy gia tốc tuyến tính?
        \item Sự khác nhau giữa xạ trị trong và xạ trị ngoài
    \end{enumerate}
\section{Trả lời câu hỏi}
Xạ trị là một phương pháp điều trị bệnh bằng cách sử dụng các tia bức xạ ion hóa chiếu vào vùng tế bào ung thư nhằm hạn chế sự phát triển cũng như tiêu diệt khối bướu đó. Ngày nay, xạ trị đang đóng một vai trò đặc biệt quan trọng trong việc điều trị ung thư.
    \begin{enumerate}
        \item %! Xa tri trong
        Xạ trị trong là:
            \begin{itemize}
                \item Phương pháp xạ trị bằng cách đặt nguồng vào trong hoặc gần khối u, cho một liều bức xạ cao tại một khoảng cách ngắn, khu trú chính xác đến các khối u trong khi giảm thiểu ảnh hưởng của bức xạ đến các mô lành gần đó.
                \item Gradient liều suy giảm rất nhanh, tỷ lệ với bình phương khoảng cách
                \item Phân loại xạ trị trong (xạ trị áp sát): Xạ trị sát bướu, xạ trị cắm lưu xuyên mô.
                \item Nguồn phóng xạ: \ce{192Ir}
            \end{itemize}
        Quy trình xạ trị trong (xạ trị áp sát):
            \begin{itemize}
                \item Chụp phim mô phỏng
                \item Lập kế hoạch xạ trị: Scan film $\rightarrow$ Nhập hệ trục toạ độ, tái tạo hình ảnh $\rightarrow$ Phân bố điểm dừng nguồn $\rightarrow$ Áp liều điểm $\rightarrow$ Điều chỉnh liều
                \item Duyệt, in chuyển kế hoạch qua máy Treatment Console để xạ trị
                \item Xạ trị
                \item Tháo bộ áp, theo dõi bệnh nhân
            \end{itemize}
        \item %! Xa tri ngoai 
        Xạ trị ngoài là
            \begin{itemize}
                \item Một phương pháp phổ biến nhất trong kỹ thuật xạ trị
                \item Sử dụng chùm photon và chùm electron mang năng lượng cao, và các chùm tia này được tạo ra từ máy gia tốc tuyến tính. 
                \item Các chùm tia photon hay tia X năng lượng cao có khả năng đâm xuyên rất lớn nên được sử dụng để điều trị các khối bướu nằm sâu bên trong cơ thể bệnh nhân. 
                \item Các chùm tia electron thường có liều bề mặt cao và sự suy giảm nhanh sau quãng chạy của nên thường được áp dụng cho các khối bướu nông gần bề mặt.
            \end{itemize}
            Quy trình xạ trị ngoài gồm các bước sau : 
                \begin{itemize}
                    \item Cố định bệnh nhân: Cố định đầu, tay chân, hoặc toàn thân
                    \item Mô phỏng: Mô phỏng theo quy ước 2D hoặc Mô phỏng CT 3D
                    \item Xác định vùng thể tích điều trị và các cơ quan lành: Xác định thể tích mô lành và thể tích điều trị
                    \item Lập kế hoạch xạ trị ngoài: Thiết lập trường chiếu, tính liều, khảo sát liều, đánh giá kế hoạch xạ trị
                    \item Kiểm tra kế hoạch xạ trị (QA): Kiểm tra liều chiếu xạ, kiểm tra trường chiếu
                    \item Thực thi kế hoạch xạ trị ngoài: Xác định lại các thông số trên phần mềm điều trị ( RT-chart ) $\rightarrow$ -	Hẹn ngày giờ điều trị cho bệnh nhân trên máy tính ( Time Planer) $\rightarrow$ Đặt bệnh nhân đúng theo tư thế lúc mô phỏng ban đầu $\rightarrow$ 	Điều khiển máy xạ trị cho bệnh nhân
                    \item Theo dõi: đánh giá thể trạng, bướu của bệnh nhân, điều chỉnh liều lượng, chỉ định xạ trị cho phù hợp
                \end{itemize}
        % \item %! Su khac nhau giua xa tri trong va xa tri ngoai
        

    \begin{sidewaystable}[htbp]
        \centering
        \caption{So sánh sự khác nhau của xạ trị trong và xạ trị ngoài}
        \begin{tabular}{|p{0.6cm}|p{2.5cm}|p{11cm}|p{11cm}|}
            \hline
            \textbf{STT} & \textbf{Tiêu chí} & \textbf{Xạ trị trong} & \textbf{Xạ trị ngoài}\\ \hline
            %-------------------------------------------------------------------------
            1& Vị trí đặt nguồn & Nguồn đặt ở trong, gần khối u & Nguồn hoặc máy xạ trị đặt ở ngoài \\ \hline
            2& Thiết bị & Máy xạ trị dùng nguồn \ce{192Ir} & Máy xạ trị tia X, Cobalt-60,  máy gia tốc tuyến tính,... \\ \hline
            %-------------------------------------------------------------------------
            3&Ưu điểm & \begin{minipage}[t]{11cm}
                \begin{itemize}
                    \item Khả năng tối ưu hoá liều
                    \item Giảm biến chứng mô lành so với thuần tuý xạ trị ngoài
                    \item An toàn bức xạ: Hạn chế việc nhân viên tiếp xúc với bức xạ; Loại bỏ các việc chuẩn bị và vận chuyển nguồn; Giảm thiểu tối đa nguy cơ mất nguồn 
                \end{itemize}   
                \vspace{0.4em}                 
            \end{minipage}
            & \begin{minipage}[t]{11cm}
                \begin{itemize}
                    \item Có thể điều trị những vị trí bướu mà phẫu thuật khó với tới được
                    \item Có thể kiểm soát bướu tại chổ ở mức độ vi thể.
                \end{itemize}
                \vspace{0.4em}
            \end{minipage} \\ \hline
            %-------------------------------------------------------------------------
           4& Nhược điểm & \begin{minipage}[t]{11cm}  
                \begin{itemize}
                    \item Chi phí mua nguồn cao, nên độ phổ biến chưa cao, nhưng vẫn đang nghiên cứu và phát triển thêm
                    \item Thời gian điều trị ngắn không cho phép việc tái tạo các tể bào tổn thương ở mô lành,.. và không sữa chữa kịp thời lỗi có thể xảy ra
                    \item Khả năng nhân viên và bệnh nhân nhân được liều cao khi có sự cố kẹt nguồn.
                \end{itemize}                  
            \end{minipage}  & 
            \begin{minipage}[t]{11cm}
                \begin{itemize}
                    \item  Chi phí điều trị cao hơn xạ trị trong 
                    \item Quy trình lập kế hoạch điều trị cần cụ thể và phức tạp hơn xạ trị trong 
                    \item Giá thành đầu tư thiết bị cao
                    \item Chi phí bảo dưỡng cao, đối với máy xạ trị gia tốc tuyến tính 
                    \item Công tác quản lý và đảm bảo an ninh nguồn tốn kém và gặp nhiều khó khăn, đối với máy xạ trị Cobalt 60
                    \item Nhân viên và người bệnh có thể nhận được suất liều cao nếu có sự cố của máy gia tốc, máy xạ trị Cobalt 60,...
                \end{itemize}
                \vspace{0.4em}
            \end{minipage}\\    \hline
        \end{tabular} 
    \end{sidewaystable}
    \end{enumerate}